\documentclass[../main.tex]{subfiles}
\begin{document}


Here we give a brief description of the Husky programming language, invented to implement the above ideas.

\subsection{Requirements for AGI-Ready Programming Language}

Here we lay down the requirements for a programming language that can implement the efficient AGI as we described.

\paragraph{Basic Requirements} We first discuss basic requirements, each of which is satisfied by at least one current programming language or is among the future design goals. However, there is no single language that satisfies all of them, a direct consequence of the fact that computer science has only been developed for 70 years.

\begin{enumerate}[(i)]
	\item purity.


We don't want to use a programming language with which we can shoot ourselves in the feet.

Logic errors is fine, but it can be detected in the debugger and handled in a controlled manner, but memory bugs and other undefined behaviors will make a large system extremely hard to track. One needs to know much more to debug a memory bug, than a logic one.

	\item modularity
	\item term level coding
	\item system level coding
	\item all level debugging support
	\item 
\end{enumerate}

\paragraph{Advanced Requirements}

\begin{enumerate} [(i)]
	\item express evolving computation graph
\end{enumerate}


\subsection{}

\subsection{}

\subsection{Type System}

\subsubsection{Concept Level Types}


\subsubsection{System Level Types}

\subsubsection{Types for Machine Learning}

\subsection{Debugger}

\subsubsection{Lazy}

\subsubsection{Eager Functional}

\subsubsection{Eager Procedural}

\subsubsection{Machine Learning}


\subsection{Visualization}

\subsection{Notebook}
\end{document}